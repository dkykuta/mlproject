\documentclass{lista}

\titulo{Projeto\\Reconhecimento do último digito de placas de automóveis}
\materia{MAC5832 - Aprendizagem Computacional:\\Modelos, Algoritmos e Aplicações}

\aluno{Diogo Haruki Kykuta}{6879613}
\aluno{Bruno Padilha}{xxxxxxx}

\begin{document}

\cabecalho

\section{O problema}
Na cidade de São Paulo, temos um rodízio municipal de veículos no qual
automóveis com placas terminadas em certos dígitos não podem trafegar em
certas regiões da cidade. Então, reconhecer o último dígito de uma
placa de carro a partir de uma foto pode automatizar a fiscalização
do rodízio, em busca de infratores. Isso torna este um estudo
interessante.

\section{Solução Proposta}
Nossa proposta de solução consiste de várias partes, podendo ser
agrupadas em dois grandes grupos:

\subsection{Treinamento}
A primeira parte da solução consiste em realizar o treinamento, para
que consigamos criar classificadores bons e os gravamos.
Precisamos de dois classificadores: um para extrair a placa de uma foto
da traseira do carro e outro para reconhecer o dígito depois de extraído.

\subsection{Predição}
Usando os resultados obtidos pelos dois treinamentos mencionados acima,
recebemos uma foto (supostamente contendo uma placa) como argumento
e o programa reconhece qual é o último dígito. Essa
informação está no programa na forma de um inteiro, mas por se tratar
apenas de um protótipo para estudo, essa informação é apenas impressa
no terminal. Mas poderia isso poderia ser usado dentro de
outra aplicação maior.

\section{Problemas Encontrados}
A detecção de placas mostrou-se um problema...

\section{Solução desenvolvida}
A linguagem de programação escolhida para o programa ser feito foi Python,
pela facilidade que a linguagem proporciona em lidar com vetores, matrizes
e também pela grande lista de bibliotecas a disposição. Dentre elas,
algumas foram usadas em nosso programa, e estão listadas a seguir:

\begin{itemize}
	\item[numpy] Essa biblioteca permite a manipulação de vetores e matrizes
		de forma mais fácil e intuitiva. Permitindo uma melhor leitura do
		código, além de uma alta performance. Todas as suas operações são
		muito otimizadas.

	\item[opencv] Usamos a versão 2 do OpenCV, cuja principal característica
		(interessante para nós) é a representação de imagens como objetos
		do numpy. Dessa forma, temos fácil acesso aos pixels da imagem, ainda
		podendo usar todo o ferramental dado pelo OpenCV. Essa biblioteca nos
		permite realizar várias operações na imagem, como operações
		morfológicas (como abertura e fechamento), limiarização,
		redimensionamento, dentre outras. OpenCV possui funções de
		aprendizagem computacional, mas optamos por escolher uma outra
		biblioteca para isso.

	\item[sklearn] Essa biblioteca nos coloca a disposição métodos de
		aprendizagem computacional otimizados, e escolhemos usar 
		\textit{Support Vector Machine (SVM)} para tratar nosso problema.
		A implementação que nos é fornecida permite múltiplas classes, o que
		é essencial para lidarmos com reconhecimento de dígito.
\end{itemize}

\subsection{Descrição dos passos do programa}

\subsubsection*{Passo 1: Extrair a placa}
A partir de uma foto como entrada.... [TODO]

\subsubsection*{Passo 2: Identificar regiões de interesse na placa}
Encontrar os dígitos

\subsubsection*{Passo 3: Recortar o último dígito}
Recortar de tamanho XXX o retangulo

\subsubsection*{Passo 4: Identificar o último dígito}
Usando SVM, com um treinamento feito previamente, identificamos o último
dígito.
A etapa de aprendizado usa um kernel RBF (\textit{Radial Basis Function}),
com parâmetros $C$ e $\gamma$ escolhidos por um passo de validação
cruzada (\textit{cross-validation}).

\section{Resultados}
[TODO: Mostrar 5-10 imagens de placas e o que o algoritmo retornou.
Delas, 1 placa vermelha e 1-2 placas detectadas errado por classificador
ruim e 1 placa detectada errada do fusquinha (6 antigo mais arrendondado).]

\newpage
\section*{Bibliografia}

\end{document}
