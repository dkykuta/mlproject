\documentclass{lista}

\titulo{Projeto\\Reconhecimento do último digito de placas de automóveis}
\materia{MAC5832 - Aprendizagem Computacional:\\Modelos, Algoritmos e Aplicações}

\aluno{Diogo Haruki Kykuta}{6879613}
\aluno{Bruno Padilha}{xxxxxxx}

\begin{document}

\cabecalho

\section{Introdução}
bla

\section{Solução Proposta}
bla

\section{Problemas Encontrados}
bla

\section{Solução desenvolvida}
A linguagem de programação escolhida para o programa ser feito foi Python,
pela facilidade que a linguagem proporciona em lidar com vetores, matrizes
e também pela grande lista de bibliotecas a disposição. Dentre elas,
algumas foram usadas em nosso programa, e estão listadas a seguir:

\begin{itemize}
	\item[numpy] Essa biblioteca permite a manipulação de vetores e matrizes
		de forma mais fácil e intuitiva. Permitindo uma melhor leitura do
		código, além de uma alta performance. Todas as suas operações são
		muito otimizadas.

	\item[opencv] Usamos a versão 2 do OpenCV, cuja principal característica
		(interessante para nós) é a representação de imagens como objetos
		do numpy. Dessa forma, temos fácil acesso aos pixels da imagem, ainda
		podendo usar todo o ferramental dado pelo OpenCV. Essa biblioteca nos
		permite realizar várias operações na imagem, como operações
		morfológicas (como abertura e fechamento), limiarização,
		redimensionamento, dentre outras. OpenCV possui funções de
		aprendizagem computacional, mas optamos por escolher uma outra
		biblioteca para isso.

	\item[sklearn] Essa biblioteca nos coloca a disposição métodos de
		aprendizagem computacional otimizados, e escolhemos usar 
		\textit{Support Vector Machine (SVM)} para tratar nosso problema.
		A implementação que nos é fornecida permite múltiplas classes, o que
		é essencial para lidarmos com reconhecimento de dígito.
\end{itemize}

\section{Resultados}
bla

\newpage
\section*{Bibliografia}

\end{document}
